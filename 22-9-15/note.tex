\documentclass[UTF8]{ctexart}

\usepackage{listings}
\usepackage{color}
\usepackage[colorlinks,linkcolor=blue]{hyperref}

\definecolor{dkgreen}{rgb}{0,0.6,0}
\definecolor{gray}{rgb}{0.5,0.5,0.5}
\definecolor{mauve}{rgb}{0.58,0,0.82}

\lstset{ %
    aboveskip=3mm,
    belowskip=3mm,
    showstringspaces=false,
    columns=flexible,
    basicstyle={\small\ttfamily},
    numbers=left,
    numberstyle=\tiny\color{gray},
    keywordstyle=\color{blue},
    commentstyle=\color{dkgreen},
    stringstyle=\color{mauve},
    breaklines=true,
    breakatwhitespace=true,
    tabsize=3
}

\title{笔记 22-9-15}
\author{李肖}

\begin{document}

% File cover
\maketitle

\section{时间复杂度}

\subsection{时间复杂度表示求法(上界)}
\begin{itemize}
    \item 去常数
    \item 去低次项
    \item 去高次项系数
\end{itemize}

\textbf{例子}

\begin{itemize}
    \item $n + 10^6 = O(n)$
    \item $n2 + 3n - 100 = O(n^2)$
    \item $n^2log(n) + n^3 - 15n = O(n^3)$
    \item $2^n = n ^ 2 = O(2^n)$
\end{itemize}

\subsection{时间复杂度确界}

\textbf{参考博客:}\href{https://blog.csdn.net/u012495579/article/details/86630074}{https://blog.csdn.net/u012495579/article/details/86630074}

上界就是存在大于 0 的数 $n_0$ 和 C,对任意的 $ n \geq n_0$ 有 $T(n) \leq c*g(n)$

\begin{itemize}
    \item $T(n) = \Theta(g(n))$ 确界
    \item $T(n) = O(g(n))$ 上界
    \item $T(n) = \Omega(g(n))$ 下界
\end{itemize}

\section{递归}

\subsection{汉诺塔问题}

\textbf{参考博客:}\href{https://blog.csdn.net/weixin_66030644/article/details/124276464}{博客链接}

\textbf{解决思路:}

想要解决该问题,即是要将64个金盘全部移动到C杆,则

要想把第64个金盘移到C杆,先要借助B杆,将63个金盘移动到B杆

要想把63个金盘移动到B杆,先要借助C杆,将62个金盘移动到C杆

于是,

要想把3个金盘移动到C杆,先要借助B杆,将2个金盘移动到B杆

要想把2个金盘移动到B杆,先要借助C杆,将1个金盘移动到C杆

\end{document}