\documentclass[UTF8]{ctexart}

\usepackage{listings}
\usepackage{color}
\usepackage{amsmath} 
\usepackage[colorlinks,linkcolor=blue]{hyperref}

\definecolor{dkgreen}{rgb}{0,0.6,0}
\definecolor{gray}{rgb}{0.5,0.5,0.5}
\definecolor{mauve}{rgb}{0.58,0,0.82}

\lstset{ %
    aboveskip=3mm,
    belowskip=3mm,
    showstringspaces=false,
    columns=flexible,
    basicstyle={\small\ttfamily},
    numbers=left,
    numberstyle=\tiny\color{gray},
    keywordstyle=\color{blue},
    commentstyle=\color{dkgreen},
    stringstyle=\color{mauve},
    breaklines=true,
    breakatwhitespace=true,
    tabsize=3
}

\title{笔记 22-10-13}
\author{李肖}
\date{2022 年 10 月 13 日}

\begin{document}

% File cover
\maketitle

\section{贪心算法}

\subsection{应用场景}
优化问题: 最长,最大,最小,最高……

\subsection{思路}
每步采用局部最优策略,到达全局最优策略

\subsection{例题}
\subsubsection{任务安排问题}

\textbf{博客解析: } \href{https://blog.csdn.net/DayOneMore/article/details/68938850}{博客链接}

\begin{lstlisting}[language=C++]
#include<iostream>
#include<algorithm>
#include<vector>
using namespace std;

//任务安排问题
struct task {
    int id;
    int start;
    int finish;
};

bool cmpfinish(task t1,task t2)
{
    return t1.finish < t2.finish;
}

int main()
{
    int n;
    cin >> n;
    task Task[20];
    vector<task> vtask;
    for (int i = 0; i < n; i++)
    {
        cin >> Task[i].id >> Task[i].start >> Task[i].finish;
    }
    //首先将任务按照结束时间非递减进行排序
    sort(Task, Task + n, cmpfinish);
    //因为第一个任务一定在最优解里面(这个可以反证法证明出来的),所以首先将第一个任务加入
    vtask.push_back(Task[0]);
    //然后看一下后面的任务,其start开始时间,有没有与前一个任务的结束时间fi重合
    int j = 0;//用来记录最新加入的任务,以便确定需要比较的finish时间。
    for (int i = j + 1; i < n; i++)
    {
        if (Task[i].start >= Task[j].finish)//需要注意等于也可以!
        {
            vtask.push_back(Task[i]);
            j = i;
        }
    }
    vector<task>::iterator it;
    for (it = vtask.begin(); it != vtask.end(); it++)
    {
        cout << (*it).id << endl;
    }
    system("pause");
    return 0;
}

\end{lstlisting}

\end{document}