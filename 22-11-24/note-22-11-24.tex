\documentclass[UTF8]{ctexart}

\usepackage{listings}
\usepackage{color}
\usepackage{amsmath} 
\usepackage{graphicx}
\usepackage[colorlinks,linkcolor=blue]{hyperref}

\definecolor{dkgreen}{rgb}{0,0.6,0}
\definecolor{gray}{rgb}{0.5,0.5,0.5}
\definecolor{mauve}{rgb}{0.58,0,0.82}

\lstset{ %
    aboveskip=3mm,
    belowskip=3mm,
    showstringspaces=false,
    columns=flexible,
    basicstyle={\small\ttfamily},
    numbers=left,
    numberstyle=\tiny\color{gray},
    keywordstyle=\color{blue},
    commentstyle=\color{dkgreen},
    stringstyle=\color{mauve},
    breaklines=true,
    breakatwhitespace=true,
    tabsize=3
}

\title{笔记 22-11-24}
\author{李肖}
\date{2022 年 11 月 24 日}

\begin{document}

% File cover
\maketitle

\section*{动态规划}

\subsection*{算法背景}

\begin{itemize}
    \item 求解优化问题 计数问题
    \item 递归、分治、贪心、强化学习
    \item 问题可以被分解,解可以用递归的方式描述
\end{itemize}

\subsection*{应用问题}

\subsubsection*{1. 斐波那契数列}

\[
    f(n)=\left\{
    \begin{aligned}
         & f(n - 1) + f(n - 2), n > 1 \\
         & 1, n \le 1
    \end{aligned}
    \right.
\]

\end{document}