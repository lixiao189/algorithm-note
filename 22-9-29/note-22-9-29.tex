\documentclass[UTF8]{ctexart}

\usepackage{listings}
\usepackage{color}
\usepackage[colorlinks,linkcolor=blue]{hyperref}

\definecolor{dkgreen}{rgb}{0,0.6,0}
\definecolor{gray}{rgb}{0.5,0.5,0.5}
\definecolor{mauve}{rgb}{0.58,0,0.82}

\lstset{ %
    language=python,
    aboveskip=3mm,
    belowskip=3mm,
    showstringspaces=false,
    columns=flexible,
    basicstyle={\small\ttfamily},
    numbers=left,
    numberstyle=\tiny\color{gray},
    keywordstyle=\color{blue},
    commentstyle=\color{dkgreen},
    stringstyle=\color{mauve},
    breaklines=true,
    breakatwhitespace=true,
    tabsize=3
}

\title{笔记 22-9-29}
\author{李肖}
\date{2022 年 9 月 29 日}

\begin{document}

% File cover
\maketitle

\section{quicksort}

\noindent
\textbf{博客教程:} \href{https://blog.csdn.net/wthfeng/article/details/78037228}{博客地址}

\vskip 0.5cm
有一点需特别注意:若以第一个元素为基准数,\textbf{在哨兵互走过程需右边的哨兵先走}。
原因很好理解,看上面过程解析就会明白:哨兵互走交换的过程就是不断排序的过程。
若右边的哨兵先走,不管走多少次,最后相遇时的那个数是小于基准数的。
这时与基准数交换,正好分为两个序列。可若是左边的先走,相遇在大于基准数上就不好办了。

\vskip 0.5cm
\noindent
\textbf{样例代码}
\begin{lstlisting}
import typing


def quick_sort(arr: typing.List[int], l: int, r: int):
    if l >= r:
        return

    tl = l
    tr = r
    while tl < tr:
        while arr[tr] >= arr[l] and tl < tr:
            tr -= 1
        while arr[tl] <= arr[l] and tl < tr:
            tl += 1

        arr[tl], arr[tr] = arr[tr], arr[tl]

    arr[l], arr[tl] = arr[tl], arr[l]
    quick_sort(arr, l, tl - 1)
    quick_sort(arr, tl + 1, r)


def main():
    arr = [8, 7, 6, 2, 2, 2, 2, 1]
    quick_sort(arr, 0, len(arr) - 1)
    print(arr)


if __name__ == '__main__':
    main()
    
\end{lstlisting}


\end{document}