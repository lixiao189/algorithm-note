\documentclass[UTF8]{ctexart}

\usepackage{listings}
\usepackage{color}
\usepackage{amsmath} 
\usepackage[colorlinks,linkcolor=blue]{hyperref}

\definecolor{dkgreen}{rgb}{0,0.6,0}
\definecolor{gray}{rgb}{0.5,0.5,0.5}
\definecolor{mauve}{rgb}{0.58,0,0.82}

\lstset{ %
    language=python,
    aboveskip=3mm,
    belowskip=3mm,
    showstringspaces=false,
    columns=flexible,
    basicstyle={\small\ttfamily},
    numbers=left,
    numberstyle=\tiny\color{gray},
    keywordstyle=\color{blue},
    commentstyle=\color{dkgreen},
    stringstyle=\color{mauve},
    breaklines=true,
    breakatwhitespace=true,
    tabsize=3
}

\title{笔记 22-10-8}
\author{李肖}
\date{2022 年 10 月 8 日}

\begin{document}

% File cover
\maketitle
\section{度量向量的相似度}

\textbf{问题: }已知有 3 个向量, 代表了评分系统中 3 个人的评分数据,
其中 A[0] 是对第一个电影的评分,A[1] 是对第二个电影的评分 ……
$$ A = [2, 4, 1, 3, 5] $$
$$ B = [1, 2, 3, 4, 5] $$
$$ C = [3, 4, 1, 5, 2] $$

求这三个向量的相似度

\vskip 0.3cm
\textbf{方法1:}

欧几里得距离
$$ dis(A, B) = \sqrt{(a_1 - b_1) ^ 2 + (a_2 - b_2) ^ 2 + \cdots (a_n - b_n) ^ 2} $$

\vskip 0.3cm
\textbf{方法2:}

求逆序对,毕竟这里的评分相当于对这 5 个电影的排名,
如果这 5 个电影在两个人的心目中的排名数据接近
那么这两个人的喜好就很一样。

\subsection{$O(n^2)$求逆序对}

\begin{lstlisting}
arr = [3, 4, 1, 5, 2]
l = len(arr)
cnt = 0
for i in range(l):
    for j in range(i + 1, l):
        if arr[j] < arr[i]:
            cnt += 1
print(cnt)
\end{lstlisting}

\subsection{$O(n*log(n))$求逆序对}

\noindent
\textbf{TIPS: }使用归并排序

\end{document}